\section{System Model}
\label{sec:system_model}

\subsection{Network Architecture}
The proposed PSWR-DRL (Power Saving Wireless Routing based on Deep Reinforcement Learning) system is built upon a sophisticated wireless sensor network architecture that emphasizes scalability and energy efficiency. At its core, the system employs a hierarchical cluster-based architecture incorporating heterogeneous node capabilities, allowing for efficient data aggregation and reduced communication overhead. The network design supports flexible scaling from small deployments of 10 nodes to larger installations of up to 100 nodes, maintaining consistent performance across all scales. To ensure reliable performance evaluation, the system utilizes static topology within individual simulation runs, while the implementation is based on industry-standard MICAz mote hardware specifications, providing a realistic foundation for energy consumption modeling and practical deployment considerations.

\subsection{State Space Design and Representation}
The intelligence of PSWR-DRL is founded on a comprehensive 9-dimensional state space representation that captures the complete dynamics of the wireless sensor network. This sophisticated state space encompasses current energy levels across all nodes, providing real-time visibility into the network's energy distribution. The representation integrates detailed network topology information, enabling informed routing decisions based on node positions and connectivity patterns. Furthermore, it incorporates transmission urgency metrics that help prioritize critical data transfers while managing network congestion through real-time status monitoring. The state space also includes temporal transmission patterns, allowing the system to learn and adapt to recurring network behaviors and optimize long-term performance.

\subsection{Deep Q-Network Architecture}
\label{subsec:dqn_arch}
The core decision-making capability of PSWR-DRL is implemented through an advanced Deep Q-Network architecture specifically optimized for wireless sensor network environments. The neural network comprises two carefully designed layers, each containing 64 neurons, striking a balance between computational efficiency and learning capacity. The network's intelligence is guided by a novel multi-objective reward function that holistically optimizes network operation by balancing multiple critical factors: energy efficiency (weighted at 40\% to prioritize power conservation), network connectivity (30\% to maintain reliable communication paths), system performance (20\% to ensure data delivery quality), and network lifetime (10\% to promote long-term sustainability). Through extensive training, the system achieves convergence in approximately 847 episodes, ultimately delivering 85\% optimal action selection in real-world operation, demonstrating both learning efficiency and practical effectiveness.

\subsection{Power Management Framework}
The power management framework of PSWR-DRL implements a sophisticated multi-modal approach that addresses energy conservation at multiple levels. The framework's sleep scheduling module employs dynamic threshold adaptation mechanisms operating within 5-9 second ranges, automatically adjusting to network conditions and data requirements. This adaptive scheduling is enhanced by node-specific operational patterns that prevent synchronized network failures and optimize energy distribution across the network. The system achieves remarkable 95\% energy savings during sleep periods while maintaining network responsiveness.

The transmission control subsystem complements the sleep scheduling through an intelligent threshold-based mechanism that significantly reduces energy waste in communication. By implementing sophisticated data relevance assessment and network condition monitoring, the system achieves an 85\% reduction in unnecessary transmissions while maintaining a high 94.8\% packet delivery ratio. This dual approach to power management ensures efficient energy utilization without compromising network performance or data reliability.

\subsection{Energy Model and Resource Management}
The energy management system in PSWR-DRL is built on a comprehensive and realistic model derived from actual MICAz mote hardware specifications. This model incorporates real-world sensor data including temperature, humidity, and voltage measurements to ensure accurate energy consumption predictions and optimal resource allocation. The system implements node-specific energy consumption patterns that allow for 0-30\% variation between nodes, preventing uniform energy depletion and extending overall network lifetime. This sophisticated energy modeling enables precise power management decisions and contributes to the system's ability to significantly extend network operational duration while maintaining reliable performance.

\subsection{Performance Evaluation Framework}
The performance of PSWR-DRL is evaluated through a comprehensive framework that quantifies improvements across multiple critical metrics. The system demonstrates a remarkable 157\% extension in overall network lifetime compared to traditional approaches, while achieving a 205\% improvement in delaying the first node death - a critical metric for maintaining network coverage and functionality. Energy efficiency optimization shows 95\% savings in power consumption, while the intelligent transmission control system reduces unnecessary communications by 85\%. These improvements are consistently maintained across different network scales, demonstrating the system's scalability and robustness.

\subsection{Implementation and Validation Framework}
The practical implementation of PSWR-DRL incorporates extensive measures to ensure reliability and real-world applicability. The system integrates real-world sensor data for testing and validation, while employing hardware-verified energy consumption models to ensure accurate performance prediction. Operational constraints are carefully considered and incorporated into the design to maintain practical feasibility. The validation process includes comprehensive statistical testing through 300 independent simulation runs, with results achieving statistical significance at p < 0.001, confirming the robustness and reliability of the improvements. This thorough validation framework ensures that the system's performance benefits are reproducible and applicable in real-world deployments.

Through this comprehensive system model, PSWR-DRL establishes a new paradigm in wireless sensor network energy management, successfully combining advanced deep learning techniques with practical power management strategies to achieve significant and measurable improvements in network performance and longevity.
